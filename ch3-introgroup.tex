%ch3-introgroup.tex

\chapter{Pengenalan Grup}

Sekarang kita mulai belajar aljabar abstrak yang sesungguhnya! Sebuah grup adalah struktur aljabar yang paling sederhana; kita ambil sembarang himpunan, berikan sebuah operasi padanya, ditambah dengan empat aturan dasar, dan melihat apa yang dapat kita deduksi darinya. Meskipun sederhana, ada tak-berhingga banyaknya kemungkinan. Grup muncul di mana saja, dan bukan hanya dalam matematika. Misalnya, sulit untuk memahami fisika dan kimia tanpa pemahaman teori grup. 

Dalam bab ini, kita akan mendefinisikan apa itu sebuah grup, dan memberikan beberapa contoh kasus. Lalu kita juga akan membuktikan beberapa sifat-sifat dasar dari grup dan subgrup.

\section{Contoh Penting}

Pada bagian berikutnya, kita akan memberikan definisi dari sebuah grup. Untuk sementara, kita akan melihat terlebih dahulu contoh yang memotivasinya.

Misalkan $A$ adalah himpunan $\{1, 2, 3\}$. Kita ingin membahas seluruh permutasi dari $A$; yaitu, seluruh cara mengurutkan bilangan $1, 2,$ dan $3$. Misalnya, kita memiliki sebuah permutasi $\sigma$, dimana $\sigma(1) = 2, \sigma(2) = 1,$ dan $\sigma(3) = 3$. Dapat kita lihat dengan mudah bahwa terdapat 6 macam permutasi, karena ada 3 pilihan untuk $\sigma(1)$, lalu 2 pilihan sisanya untuk $\sigma(2)$, setelah dua pilihan tersebut diambil, $\sigma(3)$ sudah ditentukan nilainya.

Sedikit tambahan notasi akan membantu. Misalkan kita tuliskan sebuah permutasi $\sigma$ dengan menuliskan dua baris. Unsur-unsur dari $A$ ada di baris pertama, lalu bilangan hasil permutasi dari masing-masing unsur dituliskan di baris berikutnya; yaitu, kita tuliskan 
\[ \sigma = \begin{pmatrix}
1 & 2 & 3 \\
a & b & c
\end{pmatrix}
\]
untuk permutasi $\sigma(1) = 1, \sigma(2) = b,$ dan $\sigma(3) = 3$. Sehingga permutasi yang sebelumnya kita sebut dapat dituliskan sebagai
\[\begin{pmatrix}
1 & 2 & 3 \\
2 & 1 & 3
\end{pmatrix}.
\]
Daftar lengkap dari seluruh permutasinya adalah
\[\begin{pmatrix}
1 & 2 & 3 \\
1 & 2 & 3
\end{pmatrix},
\begin{pmatrix}
1 & 2 & 3 \\
1 & 3 & 2
\end{pmatrix},
\begin{pmatrix}
1 & 2 & 3 \\
2 & 1 & 3
\end{pmatrix},
\begin{pmatrix}
1 & 2 & 3 \\
2 & 3 & 1
\end{pmatrix},
\begin{pmatrix}
1 & 2 & 3 \\
3 & 1 & 2
\end{pmatrix}, \text{ dan }
\begin{pmatrix}
1 & 2 & 3 \\
3 & 2 & 1
\end{pmatrix}.
\]

Sekarang kita diskusikan komposisi dari dua permutasi. Misalnya, jika 
\[ \sigma = \begin{pmatrix}
1 & 2 & 3 \\
2 & 1 & 3
\end{pmatrix} \text{ dan }
\tau = \begin{pmatrix}
1 & 2 & 3 \\
3 & 1 & 2
\end{pmatrix},
\]
maka bisa kita lihat jelas bahwa $(\sigma \circ \tau)(1) = \sigma(\tau(1)) = \sigma(3), (\sigma \circ \tau)(2) = \sigma(\tau(2)) = \sigma(1) = 2$ dan $(\sigma \circ \tau)(3) = \sigma(\tau(3)) = \sigma(2) = 1$. Jadi,
\[\sigma \circ \tau = \begin{pmatrix}
1 & 2 & 3 \\
3 & 2 & 1
\end{pmatrix}.
\]
(Perlu diperhatikan kalau kita menerapkan permutasi $\tau$ terlebih dahulu, lalu $\sigma$.)

Kita bisa meninjau lebih jauh sifat-sifat permutasi terhadap operasi komposisi. Saat meninjaunya, kita lalu bandingkan dengan sifat-sifat $\ZZ$ atau $\ZZ_n$, di bawah operasi penjumlahan, dimana kita sudah terbiasa dengannya.

Pertama, kita memiliki \defword{ketertutupan}{closure}. Yaitu, jika kita pilih dua permutasi atas $A$ lalu komposisikan keduanya, kita akan mendapatkan permutasi lain dari $A$. Malahan, kita sudah membuktikannya pada Dalil \ref{thm1.2}, yang menyatakan bahwa komposisi dari dua fungsi bijektif adalah juga bijektif.

Lalu \textbf{asosiativitas}; yaitu untuk sembarang permutasi $\rho, \sigma,$ dan $\tau$, kita dapatkan $\rho \circ (\sigma \circ \tau) = (\rho \circ \sigma) \circ \tau$. Kita juga sudah menemukan sebelumnya; berdasarkan Dalil \ref{thm1.2}, komposisi dari fungsi selalu asosiatif.

Kita juga memiliki permutasi \textbf{identitas}. Jika $\sigma$ adalah sembarang permutasi dari $A$, maka
\[ \sigma \circ \begin{pmatrix}
1 & 2 & 3 \\
1 & 2 & 3
\end{pmatrix} = \begin{pmatrix}
1 & 2 & 3 \\
1 & 2 & 3
\end{pmatrix} \circ \sigma = \sigma.
\]

Akhirnya, terdapat pula \textbf{invers}; yaitu, untuk setiap permutasi $\sigma$, terdapat permutasi lain $\tau$ sedemikian sehingga
\[ \sigma \circ \tau = \tau \circ \sigma = \begin{pmatrix}
1 & 2 & 3 \\
1 & 2 & 3
\end{pmatrix},
\]
permutasi identitas. Invers di sini cukup mudah untuk dihitung secara langsung; misalnya,
\[\begin{pmatrix}
1 & 2 & 3 \\
2 & 3 & 1
\end{pmatrix} \circ \begin{pmatrix}
1 & 2 & 3 \\
3 & 1 & 2
\end{pmatrix} = \begin{pmatrix}
1 & 2 & 3 \\
3 & 1 & 2
\end{pmatrix} \circ \begin{pmatrix}
1 & 2 & 3 \\
2 & 3 & 1
\end{pmatrix} = \begin{pmatrix}
1 & 2 & 3 \\
1 & 2 & 3
\end{pmatrix}.
\]
Dan keberadaan invers dijamin oleh Dalil \ref{thm1.3}.

Berdasarkan diskusi kita pada bagian \ref{sec2.4} dan \ref{sec2.5}, bisa dilihat bahwa sifat-sifat tersebut juga dimiliki oleh $\ZZ$ dan $\ZZ_n$ di bawah penjumlahan. Tetapi, perlu dicatat bahwa operasi penjumlahan adalah \textbf{komutatif}. Tidak dengan kasus ini! Misalnya, 
\[ \begin{pmatrix}
1 & 2 & 3 \\
2 & 3 & 1
\end{pmatrix} \circ \begin{pmatrix}
1 & 2 & 3 \\
1 & 3 & 2
\end{pmatrix} = \begin{pmatrix}
1 & 2 & 3 \\
2 & 1 & 3
\end{pmatrix},
\]
sedangkan
\[\begin{pmatrix}
1 & 2 & 3 \\
1 & 3 & 2
\end{pmatrix} \circ \begin{pmatrix}
1 & 2 & 3 \\
2 & 3 & 1
\end{pmatrix} = \begin{pmatrix}
1 & 2 & 3 \\
3 & 2 & 1
\end{pmatrix}.
\]
Sehingga, secara umum, $\sigma \circ \tau \neq \tau \circ \sigma$.

Permutasi-permutasi tersebut di bawah operasi komposisi memberikan contoh sebuah grup, seperti yang akan segera kita bahas. Tentu saja, tidak ada hal yang khusus tentang himpunan $A = \{1, 2, 3 \}$. Dan memang, mudah saja kita gunakan himpunan $\{1, 2, 3, ..., n \}$, untuk sembarang bilangan bulat positif $n$. Himpunan seluruh permutasi dari himpunan tersebut, di bawah operasi komposisi, disebut juga sebagai \textbf{grup simetrik} pada $n$ huruf, dan dituliskan $S_n$.


\textbf{Latihan}

\begin{exc}
Pada $S_4$, misalkan $\sigma = \begin{pmatrix} 
1 & 2 & 3 & 4 \\
3 & 1 & 4 & 2
\end{pmatrix}$ dan $\tau =  \begin{pmatrix} 
1 & 2 & 3 & 4 \\
3 & 4 & 1 & 2
\end{pmatrix}$. Hitunglah nilai-nilai berikut.
  \begin{enumerate}
  	\item $\sigma \tau$
  	\item $\tau \sigma$
  	\item invers dari $\sigma$
  \end{enumerate}
\end{exc}

\begin{exc}
Pada $S_5$, misalkan $\sigma = \begin{pmatrix} 
1 & 2 & 3 & 4 & 5 \\
5 & 3 & 2 & 1 & 4
\end{pmatrix}$ dan $\tau =  \begin{pmatrix} 
1 & 2 & 3 & 4 & 5\\
2 & 4 & 1 & 3 & 5
\end{pmatrix}$. Hitunglah nilai-nilai berikut.
  \begin{enumerate}
  	\item $\sigma \tau \sigma$
  	\item $\sigma \sigma \tau$
  	\item invers dari $\sigma$
  \end{enumerate}
\end{exc}

\begin{exc}
  Berapa banyak permutasi pada $S_n$? Pada $S_5$, berapa banyak permutasi yang memiliki nilai $\alpha(2) = 2$?
\end{exc}

\begin{exc}
  Misalkan $H$ adalah himpunan dari seluruh permutasi $\alpha \in S_5$ dimana $\alpha(2) = 2$. Sifat-sifat manakah dari yang kita diskusikan sebelumnya (ketertutupan, asosiativitas, identitas, invers) yang dimiliki oleh $H$ di bawah komposisi fungsi?
\end{exc}

\begin{exc}
  Tinjau himpunan seluruh fungsi dari $\{ 1, 2, 3, 4, 5 \}$ ke $\{ 1, 2, 3, 4, 5 \}$. Sifat-sifat (ketertutupan, asosiativitas, identitas, invers) apa sajakah yang dimiliki oleh himpunan fungsi tersebut di bawah komposisi?
\end{exc}

\begin{exc}
  Misalkan $G$ adalah himpunan dari seluruh permutasi pada $\NN$. Sifat-sifat apasajakah (ketertutupan, asosiativitas, identitas, invers) yang dimiliki oleh $G$ di bawah komposisi fungsi?
\end{exc}


\section{Grup}

Sekarang kita mulai dengan definisi umum dari sebuah grup.

\begin{defn}
  Sebuah \textbf{grup} adalah sebuah himpunan $G$, bersama dengan operasi biner, yang memenuhi syarat-syarat berikut:
  \begin{enumerate}
  	\item $a * b \in G$ untuk setiap $a, b \in G$ (ketertutupan);
  	\item $(a * b) *c = a * (b * c)$ untuk setiap $a, b, c \in G$ (asosiativitas);
  	\item terdapat sebuah $e \in G$ sedemikian sehingga $a * e = e * a = a$ untuk setiap $a \in G$ (keberadaan unsur identitas); serta
  	\item untuk setiap $a \in G$, terdapat sebuah $b \in G$ sedemikian sehingga $a * b = b * a = e$ (keberadaan unsur invers).
  \end{enumerate}
\end{defn}

Kita sebut $G$ sebagai sebuah grup \textbf{di bawah} $*$.

Unsur $e$ disebut sebagai \textbf{identitas} dari grup. Jika $a \in G$, dan $a * b = b * a = e$, maka $b$ disebut sebagai \textbf{invers} dari $a$, dan kita tuliskan sebagai $b = a^{-1}$.